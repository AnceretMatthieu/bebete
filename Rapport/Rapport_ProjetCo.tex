\documentclass[twoside]{EPURapport}
%\usepackage{listings}

%\renewcommand{\lstlistlistingname}{Liste des codes}
%\renewcommand{\lstlistingname}{Code}

%\addextratables{%
%	\lstlistoflistings
%}

%\swapAuthorsAndSupervisors

\usepackage{hyperref}
\hypersetup{colorlinks,%
            citecolor=black,%
            filecolor=black,%
            linkcolor=black,%
            urlcolor=blue}


\thedocument{Rapport projet collectif}{Application d'aide � l'identification d'insectes nuisibles}{Aide � l'identification d'insectes}

%\grade{D�partement Informatique\\ 4\ieme{} ann�e\\ 2011 - 2012}
\grade{D�partement Informatique \\ 4ieme ann�e \\ 2011 - 2012}

\authors{%
	\category{�tudiants}{%
		\name{Matthieu ANCERET} \mail{matthieu.anceret@etu.univ-tours.fr}
		\name{J�rome HEISSLER} \mail{jerome.heissler@etu.univ-tours.fr}
		\name{Julien TERUEL} \mail{julien.teruel@etu.univ-tours.fr}
		\name{Martin DEMEULEMEESTER} \mail{martin.demeulemeester@etu.univ-tours.fr}
		\name{Mickael PURET} \mail{mickael.puret@etu.univ-tours.fr}
		\name{Simon FAUSSIER} \mail{simon.faussier@etu.univ-tours.fr}
		\name{Zheng ZHANG} \mail{zheng.zhang@etu.univ-tours.fr}
		\name{Zhengyi LIU} \mail{zhengyi.liu@etu.univ-tours.fr}
	}
	\details{DI4 2011 - 2012}
}

\supervisors{%
	\category{Encadrants}{%
		\name{Gilles VENTURINI} \mail{gilles.venturini@univ-tours.fr}
	}
	\details{Universit� Fran�ois-Rabelais, Tours} 
	
	\category{Client}{% les encadrants professionnels de l'�quipe INNOPHYT
		\name{Ingrid ARNAULT} \mail{ingrid.arnault@univ-tours.fr}
		\name{Damien MUNIER} \mail{munier.damien@aliceadsl.fr}
		\name{Alexandre DEPOILLY} \mail{alexandre.depoilly@etu.univ-tours.fr}
	}
	\details{�quipe CETU INNOPHYT}
}

\abstracts{Description en fran�ais}
{Mots cl�s fran�ais}
{Description en anglais}
{Mots cl�s en anglais}

\begin{document}

\chapter{Introduction}

\chapter{Pr�sentation du projet}

\chapter{Contenu du rapport}

	\section{Analyse compl�mentaires}
		\subsection{Mod�lisation UML}
		Montrer les sch�mas UML r�alis�s � posteriori du cahier de specs (lecture/ecriture XML...).		
		
		\subsection{Fichier XML final + DTD}
		Format du fichier XML de donn�es final avec toutes les modifications apport�es tout au long du projet. 

	\section{Application PC}
	
	\section{Application mobile}


Parler des comp�tences acquises gr�ce � ce projet. \\


\chapter{Aspect gestion de projet}

	\section{Calendrier pr�visionnel et calendrier r�el}

	\section{Erreurs commises et probl�mes rencontr�s}
	+ parler des solutions mises en place
	
\chapter{Futur du projet}
Ce qu'il reste � faire \\
Les priorit�s \\
...

\chapter{Conclusion}

\annexes

Documentation doxygen \\
Documentation utilisateur (manuel d'utilisation) \\
Quelques �l�ments techniques bien pr�cis 

\end{document}